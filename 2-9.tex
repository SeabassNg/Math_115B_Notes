\section*{2/9}
  Jacobi symbol and primality tests for $n$.\\
  Let's say $n$ odd.\\
  If $n$ is prime,$p$, there are two ways to compile $(\frac{a}{n}) = (
  \frac{a}{p})$,
  \begin{enumerate}
    \item Euler's lemma, $(\frac{a}{p}) \equiv a^{\frac{p-1}{2}} \mod p$
    \item Extension of Euclidean algorithm, using quadratic residues of $p$
  \end{enumerate}
   But if $n$ is composite, only have part 2 to compute $(\frac{a}{n})$.\\
   Thus, a primality test (Solovay-Skossen).\\
   If $n$ is prime, then $(\frac{a}{n}) \equiv a^{\frac{n-1}{2}} \mod n$\\
   These could happen even if $n$ is not prime for some $a$.\\
   If it happens (and $n$ is composite) $n$ is an "Euler pseudoprime".\\
   \underline{Fact}: If $a^{\frac{n-1}{2}} \equiv (\frac{a}{n}) \mod n$, then
   $a^{n-1} \equiv (\frac{a}{n})^2 \equiv 1 \mod n$.\\
   Euler psuedoprime $\Rightarrow$ vanilla a-psuedoprime.\\
   Miller's Test is stronger than Euler's test.\\
   \begin{theorem}
    If $a$ is a Miller $a$-pseudoprime, then $a$ is a Euler $a$-pseudoprime.
   \end{theorem}
   Let $2^j | n - 1$ such that $\frac{n-1}{2}$ is odd.\\
   Then, if $n$ is prime, $a^{\frac{n-1}{2^j}}, a^{\frac{n-1}{2^{j-1}}} \ldots
   a^{n-1}$ is either all 1s or $\ldots, -1, 1$.\\\\
   The strategy for either test is chosen at random to make $n$ either composite
   for sure or probably prime.\\
   \begin{theorem}(Miller-Rabin)
    At least $\frac{3}{4}$ of reults reveal $n$ to be completely by Miller's 
    test. 
   \end{theorem}
   \begin{theorem}(Solovay-Strassen)
    At least $\frac{1}{2}$ of reults reveal $n$ to be completely by Euler's 
    test. 
   \end{theorem}
   \begin{theorem}(Baby Solovay Strassen)
    If $n$ is composite, $\exists a \perp n$ such that $a^{{n-1}} 
    \not\equiv (\frac{a}{n}) \mod n$
   \end{theorem}
   \begin{proof}
    say it wasn't so for some $n$.\\
    $a^{n-1} \equiv 1 \mod n$ $\forall a \perp n$, so $n$ is Carmichael.\\
    $n$ is square-free and $p-1 | n-1$ when $p | n$\\
    Say $p | n$, we supposed $a^{\frac{n-1}{2}} \equiv \pm 1 \mod p$.\\
    Let $k = \frac{n-1}{p-1}$.\\
    $a^{\frac{p-1}{2}} \equiv \pm 1 \mod p$\\
    $(a^{\frac{p-1}{2}})^k \equiv a^{\frac{n-1}{2}}$ and $k$ is odd because
    $a^{{p-1}{2}}$.\\
    So, where we are, $a^{\frac{n-1}{2}} \equiv a^{p-1}{2} \mod p$ when $p |n$\\
    By hypothesis, $a^{\frac{n-1}{2}} = (\frac{a}{n})$ and $(\frac{a}{p})$
    by fact.\\
    Therefore, $(\frac{a}{n}) = (\frac{a}{p}) \Rightarrow$ If $p,q | n$,then
    $(\frac{a}{p}) = (\frac{a}{q})$. This can't happen because $a$ is $\mod p$
    of a $\mod q$.
   \end{proof}
   \underline{Example}: Let $n = 561 = 3 \times 11 \times 17$. If this passed
    Euler's test, then $\forall a \perp n$, then 1) $n-1 = (p-1)k$ $\forall
    p | n$ and 2) $(\frac{a}{p}) = (\frac{a}{q})$.

  \subsection*{Rational and irrational numbers}
    We'll be doing this in $\mathbb{Q} \subseteq \mathbb{R}$ and in 
    $\mathbb{Q}_p$.\\
    \begin{theorem}
      $\alpha \in \mathbb{R}$ is rational iff it's a repeating decimal.
    \end{theorem}
    \begin{proof}
      Depends on how long division works. The algorithm determines by the
      remainder. If the remainder ever repeats, when you're pulling down 0s, 
      then the whole algorithm repeats after that.\\
      Repeats must happen because there is only finite choices of remainders.
      \\\\
      The converse.\\
      If digits of $\alpha$ repeats, then $\alpha = \frac{a}{b}$.\\
      2 cases:\\
      1) Terminating decimals: Yes! Ex) 7.215 = 7215 / 1000
      2) Repeating from start. $.99999999\ldots = 1$, $.001001 \ldots = 
      \frac{1}{999}$, and $.237237\ldots = \frac{237}{999}$
    \end{proof}
