\section*{2/18}
  \subsection*{Midterm question 1}
    $\ldots 777000_{10} = 1000 * \ldots 777_{10}$\\
  \subsection*{$p$-addic/$n$-addic convergence}
    of sequences or series\\
    $\lim_{k \to \infty} x_k = x$ and $x_k, x \in \mathbb{Z}_p$ or 
    $\mathbb{Z}_n$ means $\lim_{k \to \infty} d(x_k, x) = 0$ where
    $d(a,b) = 2^{-l}$ where $p^l$ or $n^l | a - b$ and $p^{l + 1}$ or
    $n^{l + 1} \not| a - b$.\\\\
  \underline{Example}:\\
    $a = 2375008$\\
    $b = 1375008$\\
    $d_{\mathbb{R}}(a,b)= 10^6 = $ a lot\\
    $d_{10}(a,b) = 2^{-6} = $ small. It converges 10-addicly.\\
  \subsection*{Geometryic series formula}
    $\sum_{k = 0}^{\infty} ba^k = \frac{b}{1 - a}$\\
    Why don't we just test it out in 10-addictly?\\
    $\sum_{k = 0}^{\infty} 7*10^k = \frac{7}{1 - 10} = -\frac{7}{9}$, which
    is what it is! 10-adically right n-addic criterion is $d_n(a,0) = a$ 
    $\forall a$.\\
  \begin{theorem}: $e$ is irrational where $e = 2 + \frac{1}{2} + \frac{1}{6}
  + \ldots = \sum_{n = 0}^{\infty} \frac{1}{n!}$
  \end{theorem}
  \begin{proof}
    Use "factorial base":
    $xxxxxxxxxxxxx.xxxxxxxx\ldots$\\
    Everything to the left of the decimal point is base 10.\\
    First place right of decimal point is base 2, next is base 3, next is
    base 4.\\
    On the other hand, $e = 2.11111\ldots = 2 + \frac{1}{2} + \frac{1}{6} + 
    \frac{1}{24} + \ldots$\\
    On the other hand, if $\frac{a}{b}$ is rational, then $\frac{a}{b} = 
      \frac{c}{b!}$ which terminates. e.g. $\frac{1}{5} = \frac{24}{120} = 
      .0104_{fac}$.\\
    Assume that an estimate. Let $\alpha \in \mathbb{R}$ and consider 
    approximating $\alpha$ by $\frac{a}{b}$ except for $\alpha = \frac{a}{b}$\\
    $\alpha \approx \frac{a}{b}$, but $\alpha \not=\frac{a}{b}$\\
  \end{proof}
  \begin{theorem}
    If $\alpha \in \mathbb{Q}$, then $|\alpha - \frac{a}{b} > \frac{c}{b}$
    depends on $\alpha$, but not $b$.\\
    $|\alpha - \frac{a}{b}| = $ small with $b$ not too big is called diophantine
    approximation.\\
    In our case, $| \alpha - \frac{a}{b}| = \Omega(\frac{1}{b})$ where
    $\Omega(f(n))$ means $\ge cf(n)$.
  \end{theorem}
  \begin{proof}
    $\alpha = \frac{x}{y}$
    \begin{eqnarray*}
      |\frac{x}{y} - \frac{a}{b}| & = & \frac{|xb - ay|}{yb}\\
      &\ge& \frac{1}{yb}\\
      & = & \Omega(\frac{1}{b})
    \end{eqnarray*}
  \end{proof}
  Back to the proof
  \begin{proof}
    How well does $e$ work in $|\alpha - \frac{a}{b}|$?\\
    To well\\
    $S_n = \sum_{k = 0}^{n} \frac{1}{k!} = \frac{a_n}{n!}$ is a close
    rational.\\
    $e - s_n| = \frac{1}{(n+1)!} + \frac{(n+2)!} + \ldots < \frac{2}{n-1}! 
    \not= \Omega(\frac{1}{n!})$, so $e \not\in \mathbb{Q}$.
  \end{proof}
  \begin{theorem}
    If $\alpha$ is algebraic of degree $n$, then $|\alpha - \frac{a}{b}|
     = \Omega(\frac{1}{b^n})$
  \end{theorem}
