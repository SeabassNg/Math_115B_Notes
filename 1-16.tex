\section*{1/16}
  Ordinary numbers in $\mathbb{R}$ have finitely many digits to left,
  infinitely many digits to the right, carry rules for $+, -, \times$,
  optional minus sign, and different bases, $b$ represent the same 
  $\mathbb{R}$.\\
  We know that $\pi = 3.14159 \ldots$\\
  \underline{Definition}: If $p$ is prime, $p$-adict number has infinitely
  many base $p$ digits to the left and finitely many to the right, and carries
  to the left set of all of those us written $\mathbb{Q}_p$, p-adic numbers\\
  Those that have no digits to the right are $p$-adic integers, 
  $\mathbb{Z}_p$\\
  \underline{Note}: $\mathbb{Z}_p \not= \mathbb{Z}/p$. $|\mathbb{Z}_p|$ is
  uncountable. $|\mathbb{Z}/p| = p$.\\\\
  \underline{Example}: $\mathbb{Z}_5$\\
  $\ldots0000_5 = 0$\\
  $\ldots0000034_5 = 34_5  = 19$\\
  Let's say I have $\ldots4444_5$ + $\ldots 0_5$ = $\ldots 0_5$. that means
  that $\ldots444444 = 0$.\\
  Can subtract anything from $\ldots000_5$ in $\mathbb{Z}/p$\\
  Therefore, minus signs are optional.\\
  Another definition/explanation of $x \in \mathbb{Z}_p$ is that it's
  any consistant sequence of residues mod $p^k$ as $k \to \infty$\\\\
  \underline{Example}: I want $x \equiv 2 \mod 5, \equiv 22_5 \mod 25 \equiv 
  12, 222_5 \mod 125 \equiv 62$.\\
  $x \in \mathbb{Z}$ is sometimes. $\mathbb{Z} \subseteq \mathbb{Z}_p$\\
  This $x \equiv -\frac{1}{2}$ because $x \equiv -\frac{1}{2} \mod 5^k$
  $\forall k$\\
  Jesus said $x^2 \equiv -1 \mod 5^k$ has two solutions for all $k$.\\
  Therefore, it has two 5-adic solutions, so $x = \ldots 12_5$ solutions
  to $x^2 = -1 \in \mathbb{Z}_5$\\
  \begin{theorem}
    $\mathbb{Z}_p$'s are all inequivalent rings, all inequivalent to 
    $\mathbb{R}$ or $\mathbb{C}$
  \end{theorem}
  \underline{Example}: $\mathbb{Z}_5 \not\equiv \mathbb{R}$ because $i \in
    \mathbb{Z}$
  \begin{theorem}
    $\mathbb{Q}_p$ is closed under division
  \end{theorem}
  Actually, everything $\mathbb{Z}_p$ can divide by any $x$ such that $p 
  \not| x$.\\
  Can make $\mathbb{Z}_n$ for composite $n > 1$ too, but they don't add much
  to $\mathbb{Z}_p$'s.
