\section*{1/30}
  Quadratic reciprocity thm:\\
  1st criterion and definition\\
  "$a$ on $p$" $(\frac{a}{p}) = \begin{cases} 1 & \text{ if there exists $x$, a 
  solution to $x^2 \equiv a^e$}\\ -1 & \text{ if not} \\ 0 & \text{ if p | a} 
  \end{cases}$\\
  2nd criterion is Euler's lemma: $(\frac{a}{p}) \equiv a^{\frac{p-1}{2}} \mod 
  p$\\
  3rd criterion is Gauss' lemma: $(\frac{a}{p}) = (-1)^s$ where $s$ is the
   number of left-half residues, $1 \le j \le \frac{p-1}{2}$, such that
   $a_j$ is the right half, $\frac{p+1}{2} \le a_j \mod p \le p-1$\\\\
 \begin{proof} 
  Compare two products mod $p$.\\
  $x \equiv 1 \times 2 \times \ldots \times \frac{p-1}{2} \mod p$\\
  $y \equiv a \times (2a) \times (3a) \times \ldots \times ((\frac{p-1}{2})a)
  \mod p$\\
  $\frac{y}{x}$ is, 1st of all, defined on mod $p$ ($p \not| x$).\\
  $\frac{y}{x} \equiv a \times a \times \ldots \times a \equiv 
  a^{\frac{p-1}{2}} \equiv (\frac{a}{p})$\\
  $s$ is the number of left-half $j$'s with $a_j$ is the half.\\
  Let $G = \{ a, 2a, \ldots, \frac{p-1}{2} a\}$, all of which has a different
  equivalence mod $p$.\\
  $|G| = \frac{p-1}{2}$. No two elements of $G$ are negatives either.
  $a_j \equiv -ak$, then $j \equiv -k$ can't happen if $j,k$ are left half.\\
  this happens $s$ times, so $\frac{y}{x} \equiv (-1)^s$ also.\\
  $(\frac{a}{p} \equiv \frac{y}{x} \equiv (-1)^s$
 \end{proof} 
 \underline{Example}: $a = 3$, $p=11$.\\
 $x \equiv 1 \times 3 \times 3 \times 4 \times 5$\\
 $y \equiv 3 \times 6 \times 9 \times 1 \times 4 \equiv 3 \times (-5) \times 
 (-2) \times 1 \times 4 \equiv x \times (-1)^s$ with $s = e$ in this case.\\
 \begin{theorem}
  Using Gauss's Lemma,\\
  $(\frac{2}{p}) = \begin{cases} 1 & \text{ if $p \equiv 1$ or $7 \mod 8$}\\
  -1 & \text{otherwise} \end{cases}$\\
  $(\frac{-1}{p}) = \begin{cases} 1 & \text{ if $p \equiv 1 \mod 4$}\\
  -1 & \text{if $p \equiv 3 \mod 4$} \end{cases}$\\
 \end{theorem}
 \begin{proof}
  $(\frac{-1}{p}) = (-1)^s$ where $s$ is $s = \frac{p-1}{2}$ in this case $=
  (-1)^{\frac{p-1}{2}}$\\\\
  $(\frac{2}{p}) = (-1)^s$ \\
  i.e. $(\frac{2}{11}) =$ number of $1 \le j \le \frac{p-1}{2}$, such that
  $2j | \frac{p-1}{2}$ and $2j < p$\\
  Let $s$ be the number of $\frac{p}{2} < 2j < p$ where $p$ is odd.\\
  Then, the number of $2j < p$ is $\frac{p-1}{2}$ which is even if $p \equiv 1
  \mod y$ and odd if $p \equiv 3 \mod 4$\\
  Then, the number of $2j < \frac{p-1}{2}$ is $\lfloor\frac{p-1}{4}\rfloor$ 
  which is even if $p \equiv  \text{odd} \mod 8$ and odd otherwise.\\
  $s \equiv 0$ if $p \equiv 1 \text{ or } 7 \mod 8$ and $1$ if $p \equiv 
  3 \text{ or } 5 \mod 8$
 \end{proof}
 \begin{lemma}: Gauss Lemma\\
    $(\frac{a}{p}) = (-1)^s$ where $s \equiv \sum_{j=1}^{p-1} \lfloor
    \frac{2a_j}{p} \rfloor \mod 2$
 \end{lemma}
