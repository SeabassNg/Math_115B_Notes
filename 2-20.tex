\section*{2/20}
  \subsection*{Irrational and transcendental numbers}
    $\alpha \in \mathbb{R}$ is algebraic means $f(\alpha)= 0$ for some 
    polynomial with integer coefficients.\\
    This generalizes root constructions.\\
    \begin{theorem}
      If $f(\alpha) = 0$ where coefficients of $f$ are algebraic, then
      $g(\alpha) = 0$ where coefficients are integers.
    \end{theorem}
    \underline{interesting ideas}: $\pi$ is not algebraic. Even though it's
    from a circle, you use calculus to get the perimeter of the circle and
    areas of the circle.\\
    $\alpha$ is transcendental if it is not algebraic.\\
    \begin{theorem}
      Transcendental exists.
    \end{theorem}
    \begin{theorem}
      $e$ and $\pi$ are transcendental.
    \end{theorem}
    Proving $e$ and $\pi$ are irrational is very difficult, so we won't...\\
    \begin{theorem}
      $2^{\pi}$ is irrational and transcendental.
    \end{theorem}
    No one can prove this.\\
    Let's say that $\alpha = \frac{a}{b}$. Then, $f(\alpha) = 0$ where
    $f(\alpha) = a - bx$, so rational numbers are algebraic.\\\\
      Proof 1 of transcendental existence: Set $A$ of algebraic numbers is 
      countable, but $\mathbb{R}$ is uncountable.\\
    Computer Science approach to countability:\\
      If $S$ is a set of elements described by finite words by a finite 
      alphabet, then it's countable.\\
      Order $s$ as words of length 1 in alphabetical order, then words of 
      length 2 in alphbetical order, $\ldots$.\\
      If $x \in S$ has many names, just use the first one.\\
      \underline{Example}: $\mathbb{Q}$ is countable.\\
        Alphabet: 0, 1, $\ldots$, 9, /, -.\\
        Just use lowest term.\\
        My dictionary: 0, 1, 2, 3, $\ldots$, 9, 10, 11, $\ldots$, 99, -1, 
        $\ldots$, -9, 100, $\ldots$, 199, 1/2, $\ldots$, 1/9, 200, $\ldots$,
        $\ldots$\\\\
      $\alpha \in \mathbb{R}$, algebraic, described by a word with alphabet:
      0, 1, $\ldots$, 9, x, +, -, ;\\
      For the polynomials, $\sqrt{2}$ is described by $xx - 2; 2$ where $;$ is
      the separator. The last 2 is the kth root from smallest to largest.\\
      $xx - 2;2$ means 2nd root of $x^2 - 2 = 0$.\\\\
    Proof 2: If $\alpha \in \mathbb{Q}$, $|\alpha - \frac{a}{b}| > \frac{c}{b}
     = \Omega(\frac{1}{b})$ when $\alpha \not= \frac{a}{b}$\\
     Let $\alpha = \frac{x}{y}$. $|\frac{x}{y} - \frac{a}{b}| = \frac{|xb - ya|}
     {by} \ge \frac{1}{by}$\\
     \begin{theorem}
      If $\alpha$ is algebraic number of degree $n$, $|\alpha - \frac{a}{b}|
      \ge \frac{c}{b^n}$, so if we choose $\alpha$ very close to $\frac{a}{b}$,
      it's transcendental.
     \end{theorem}
     \begin{theorem}
      $x = .1100010000\ldots010\ldots01$\\
      Say fourth 1 is on the 24th digit and fifth one is at 120th.\\
      Well... $x = \sum_{n = 1}^{\infty} 10^{-n!}$.\\
      Let $x = \frac{a}{b}$ where $b = 10^{n!}$.\\
      Then, $|\alpha - \frac{a}{b}| \approx 10^{-(n+1)!} \le 2 \times 
      10^{-(n+1)!}$\\
      $|\alpha - \frac{a}{b}| \ge \frac{c}{b^n}$
     \end{theorem}
    \begin{proof}
      Idea is, if $f(x)$ is a polynominal of degree $n$, with coeffcients $\in
      \mathbb{Z}$ can bound $|f(\frac{a}{b})|$\\
      $f(x) = x^3 - x + 1$, $|\frac{a^3}{b^3} - \frac{a}{b} + 1| = 
      \frac{integer}{b^3} \ge \frac{1}{b^3}$.\\
    \end{proof}
