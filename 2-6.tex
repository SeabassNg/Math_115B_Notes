\section*{2/6}
  Why $(\frac{a}{b})(\frac{b}{a}) = (-1)^{\frac{a-1}{2}\frac{b-1}{2}}$
  when $a,b$ are odd and positive.\\
  \begin{enumerate}
    \item Both sides are bimultiplicative. The left side comes from defintion.
      The right side are cases $\mod 4$
    \item Equal when $a,b$ are prime, quadratic reciprocity
  \end{enumerate}
  Likewise, $(\frac{-1}{n}) = (-1)^{\frac{n-1}{2}}$ and 
  $(-1)^{\frac{n^2-1}{8}}$\\
  \begin{enumerate}
    \item Both sides are completely multiplicative
    \item = when $n$ is prime $\Leftarrow$ Gauss' Lemma, and in $(\frac{a}{n})$,
    $a$ only matters $\mod n \Rightarrow$ algorithm to compute Jacobi symbol
    $\Rightarrow$ Legendre symbol
  \end{enumerate}
  According to the book $(\frac{a}{n})$ is defined when $n > 0$, $a < 0$, or
  $a > 0$\\
  I disagree. Let $(\frac{a}{-1}) = 1$. Then, $(\frac{a}{-n}) = (\frac{a}{n})$
  to make all of the laws still work.\\\\
  \underline{Example}: $(\frac{23}{101}) = (\frac{101}{23}) = (\frac{9}{23})$\\
  The second step is bcause 101 is $1 \mod 4$ and so is 9.\\
  Now, $(\frac{9}{23}) = (\frac{23}{9}) = (\frac{5}{9}) = (\frac{9}{5}) = 
  (\frac{4}{5}) = (\frac{-1}{5}) = (-1)^{\frac{s-1}{2}} = 1$\\\\
  $(\frac{a}{b})$ what if $a$ or $b$ is even?\\
  $(\frac{26}{101}) = (-\frac{75}{100}) = (\frac{-1}{100}) (\frac{72}{100}) =
    \ldots$\\
  Also, $(\frac{26}{100}) = (\frac{2}{101})(\frac{13}{1001}) = 
    -(\frac{13}{101}))$\\
  Since $(\frac{a}{b}) = \pm \frac{b}{a}$ when both odd, can reduce a mod b
  in $(\frac{a}{b})$, and $(-\frac{a}{b}) = (\frac{-1}{b})(\frac{a}{b})$ and
  $(\frac{2}{10}$ is known and $(\frac{2a}{b}) = \frac{2}{b} \Rightarrow$
  a version of Euler's algorithm\\\\
  Miller-Rabin is Miller's test choosing at random. It exposes that $n$ is
  composite of at least 3 quarters of the time. Then, $n$ is probably prime
  or is is definitely composite.\\\\
  Special cases: Fermat numbers/primes.\\
    Is $2^n + 1$ prime? If it is, either $n = 0$ or $n = 2^k$\\\\
  Mersenne primes: Primes that are $2^n - 1$\\\\
  \begin{theorem}
    If $f_k = 2^{2^k} + 1$ as fermat and $k > 0$.\\
    Then, $a = 2$ is not interesting.\\
    Then, $a = 3$ exposes composite $f_k$ for sure.\\
    Then, in fact, $3^{\frac{f_k - 1}{2}} \equiv -1 \mod f_k$ iff $f_k$ is
    prime.
  \end{theorem}
  Why is $3^{\frac{f_k - 1}{2}} \equiv -1$ when $f_k$ is prime. (if direction)
  \begin{proof}
    \begin{eqnarray*}
      3^{\frac{f_k-1}{2}} &=& (\frac{3}{f_k}) \text{ ( by euler's lemma)}\\
      & = & (\frac{f_k}{3}) \text{ (Quadratic reciprocity)}\\
      & = & (\frac{2^{2^k} + 1}{3}) \\
      & = & (\frac{4^{2^{k-1}}}{3})
    \end{eqnarray*}
  \end{proof}
