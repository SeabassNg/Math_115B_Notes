\section*{2/27}
  \subsection*{Continued fractions}
    \underline{Example}:\\
    $$
      [1; 1, 1, 1, 1, \ldots]
    $$
    Estimates:
    \begin{eqnarray*}
      1 + \frac{1}{1} & = & 1\\
      1 + \frac{1}{2} & = & \frac{3}{2}\\
      1 + \frac{1}{\frac{3}{2}} & = & \frac{5}{3}\\
      1 + \frac{1}{\frac{5}{3}} & = & \frac{8}{5}\\
      & \vdots &
    \end{eqnarray*}
    This follows the fibonnaci sequence.\\
    $$
      1 + \frac{1}{\frac{a}{b}} = \frac{b+a}{a}
    $$
    Let's derive the limit independently.\\
    If $[1;1, 1, 1, \ldots] = x$, then
    $x = 1 + \frac{1}{x}$ and $ x > 1$, then $x^2 = x + 1$.\\
    Therefore, $\frac{1 + \sqrt{5}}{2}$\\\\
    Two important lessons from the example:
    \begin{enumerate}
      \item Always fibbonacci like recurrence for numerical denomination of
        $C_k$
      \item If continued fraction of $c$ is periodic, then it is a quadratic
        equation for $x$.
    \end{enumerate}
    \begin{definition} $[a_0, a_1, \ldots]$ is a \underline{simple continued 
      fractions} means $a_k \in \mathbb{Z}$. You can write it down for any 
      values.\\
    \end{definition} 
   \underline{Example}: Not simple
   $$
    e + \frac{1}{\pi + \frac{1}{\sqrt{2}}} = [e; \pi, \sqrt{2}]
   $$
   \begin{lemma}
    (Fibbonaccish theorem) Let $s_{-1} = 1$, $s_0 = a_0$, $s_k = a_ks_{k-1} +
    s_{k-2}$, $t_{-1} = 0$, $t_0 = 1$, $t_k = a_kt_{k-1} + t_{k-2}$ where
    $k \ge 1$. Then, $C_k = [a_0; a_1, \ldots, a_k] = \frac{s_k}{t_k}$.\\
    $C_k$ may not be a simple continued fraction and the fraction is
    sustanominal fraction.
   \end{lemma}
   \begin{theorem}
    If $[a_0; a_1, \ldots, a_k]$ is simple, $\frac{s_k}{t_k}$ is lowest term
    and $s_k$ and $t_k$ are integers.\\
   \end{theorem}
   \begin{proof}
    By induction on $k$.\\
    Suppose we know this for some $k$.\\
    $$
      C_{k+1} = [a_0, a_1, \ldots, a_{k +1}] = [a_0, a_1, \ldots, a_{k - 1}, a_k 
      + \frac{1}{a_{k+1}}]
    $$
    Let $C_k'$ be the $k$th term of in the continued fraction.\\
    \begin{eqnarray*}
      C_k' & = & \frac{a_k' s_{k-1} + s_{k-2}}{a_k' t_{k-1} + t_{k-2}} = \frac{s_k'}{t_k'}\\
      & = & \frac{(a_k + \frac{1}{a_{k+1}})s_{k-1} + s_{k-2}}{(a_k + \frac{1}
      {a_k+1})t_{k-1} + t_{k-2}}\\
      & = & \frac{a_ks_{k - 1} + s_{k - 2} + \frac{s_{k-1}}{a_{k+1}}}{a_kt_{k - 1} + t_{k - 2} + \frac{t_{k-1}}{a_{k+1}}}\\
      & = & \frac{a_{k+1}s_k + s_{k - 1}}{a_{k+1}t_k + t_{k-1}}\\
      & = & \frac{s_{k+1}}{t_{k+1}}
    \end{eqnarray*}
    We just need the base case and we proved this.
   \end{proof}
   \begin{theorem}
    $$
      s_kt_{k-1} - s_{k - 1}t_k = (-1)^{k-1}
    $$
    if $[a_0, a_1, \ldots, a_k]$ is simple.
   \end{theorem}
   From this theorem, we know that $gcd(s_k, t_k) = 1$ where $gcd{s_k}{t_k} = 
   1$, $gcd(s_k, s_{k-1}i_ = 1$, $gcd(t_k, t_{k-1}) = 1$.\\
   The proof of this theorem is done by induction too.\\\\
   One of the goals we're after, all simple continued fractions converge. We
   need the following lemma to prove it.\\
   \begin{lemma}
    $$
      c_0 < c_2 < c_4 < \ldots < \ldots < c_5 < c_3 < c_1
    $$
    always happens in simple continuous fractions.
   \end{lemma}
   The proof of this lemma is done with the previous theorem.
