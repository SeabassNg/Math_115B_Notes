\section*{1/23}
  \subsection*{More Mobius inversions}
  $f(n)$ is some function and $F(n)$ is the $f(n)$'s summatory function.\\
  $F(n) = \sum_{d|n}f(d)$\\
  You can also find $f(n)$ from $F(d)$ too. $f(n) = \sum_{d|n} \mu(d)F(\frac{n}
  {d}) = \sum_{d|n} \mu(\frac{n}{d}) F(d)$\\
  Most important point: You have a bijection between $f$ and $F$.\\\\
  \underline{Example}: Last quarter, we learn the following:\\
    \begin{itemize}
      \item $\tau(n)$ is the number of divisors of $n$
      \item $\sigma(n)$ is the sum of divisors of $n$
      \item $\phi(n)$ is the number of prime residues $\mod n$
    \end{itemize}
    $\tau(n)$ is the summatory function of $f(n) = 1$ $(\tau(n) = \sum_{d|n}1)$
    , so $1 = \sum_{d | n} \mu(\frac{n}{d}) \tau(d)$\\\\
    $\sigma(n) = \sum_{d |n} d$, so $\sigma(n)$ is the summatory function of 
    $f(n) = n$ making $n = \sum_{d|n} \mu(\frac{n}{d})\sigma(d)$.\\\\
    What about $\phi(n)$?
    \begin{theorem}
      The summatory function of $\phi(n)$ is $F(n) = n$, so
      $\phi(n) = \sum{d|n}\mu(\frac{n}{d)})$
    \end{theorem}
    \underline{example and idea}: Let $n = 10$. The plan is to assemble
    $\mathbb{Z}/10$. From $\mathbb{Z}/d)^x$ for $d | a$.\\
    \begin{proof}
      If $a \in \mathbb{Z}/n, gcd(a,n) = d$ for $d | n$ \\
      Segregate $a$'s by this value. For each of them, there are 
      $f(\frac{n}{d})$ choices for $a$ because $a = db$, $b \in
      (\mathbb{Z}/\frac{n}{d})^x$, so $n = \sum_{d|n} \phi(\frac{n}{d})
      = \sum_{d|n}\phi(d)$
    \end{proof}
    \begin{theorem}{Primitive root theorem}
      If $p$ is prime, then $\mathbb{Z} / p$ has a primitive root, $r$, an
      element of order $p-1$.\\
      Then, the elements of $\mathbb{Z}/p$ are powers of $r$.\\
      Then, there are $\phi(p-1)$ primtive roots, $r^x$ where $x \in
      (\mathbb{Z}/(p-1))^x$.
    \end{theorem}
    Actually, we'll go straight to this.\\
    What we need for this proof:
    \begin{enumerate}
      \item Lagrange's theorem: We have $\le$ degree of $a(x)$ solutions to 
        $a(x) \equiv 0 \mod p$\\
      \item Number of solutions to $x^{p-1} \equiv 1 \mod p$ is $p-1$
      \item The defintion of $f_p(n)$ and $F_p(n)$
      \item The lemma that will be introduced later
    \end{enumerate}
    \begin{definition}
      \begin{itemize}
        \item $f_p(n)$ is the number of residues of order $n \mod p$
        \item $F_p(n)$ is the number of solutions to $x^n \equiv 1 \mod p$
          where $ord_p x | n$
      \end{itemize}
    \end{definition}
      If you let $d = ord_px$, definitions give us $F_p(n) = \sum_{d|n}
        f_p(d)$\\
      Our goal is to show $(\mathbb{Z}/p)^x$ are a big circle, i.e.
      \begin{eqnarray*}
        F_p(n)  =  n \text{when }n | p-1 & \Leftrightarrow &
        f_p(n) = \phi(n) \text{ when } n | p-1\\
        & \Leftarrow & \text{Today's lemma}\\
        & \Rightarrow & \text{Mobius inversion}
      \end{eqnarray*}
    \begin{lemma}
      If $n | p-1$, then $x^n \equiv 1 \mod p$ has exactly $n$ solutions.
    \end{lemma}
    \begin{proof}
      $x^{p-1} - 1 = (x^n -1) \times a(x)$\\
      On the left side, we have $p-1$ solutions.\\
      $x^n - 1$ has at least $n$ solutions.\\
      degree of $a(x)$ is $p-1-n$\\
      We know that $x^{ab} - 1 = (x^a - 1)(x^{ab-a} + x^{ab-2a} + x^{ab - 3a}
       + \ldots + x^a + 1)$\\
    \end{proof}
    Recall, $a$ is a quadratic residue $\mod p$ means that it's a square
    $\mod p$.\\
    \begin{definition}
      $\left(\frac{a}{p}\right) = \text{Legendre symbol} = \begin{cases}
      1 & \text{ if $a$ is a quad residue and $a \not\equiv 0 \mod p$}\\
      -1 & \text{ if $a \not\equiv 0 \mod p$ and not quad residue}\\
      0 & a \equiv 0 \mod p
      \end{cases}$
    \end{definition}
