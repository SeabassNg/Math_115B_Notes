\section*{2/25}
  \subsection*{Continued Fraction}
    A nice corollary to continued fraction
    \begin{theorem}
      (by Dirchlet)\\
      If $\alpha \not\in \mathbb{Q}$ and $\alpha \in \mathbb{R}$, then
      $| \alpha - \frac{a}{b}| < \frac{1}{b^2}$ for infinitely many
      $\frac{a}{b}$
    \end{theorem}
    \begin{theorem}
      If $\alpha \in \mathbb{Q}$, then $\exists C$ (depends on $\alpha$ only)
      such that $|\alpha - \frac{a}{b}|$ either is 0 or greater than $\frac{C}
      {b}$.
    \end{theorem}
    \noindent\underline{Examples}: $|\pi - \frac{22}{7}| = .00012\ldots < 
    \frac{1}{49}$\\
    $|\sqrt{2} - \frac{3}{2}| = .08\ldots < \frac{1}{4}$, not good enough\\
    $|\sqrt{2} - \frac{7}{5}| = .014\ldots < \frac{1}{25}$.\\

    \noindent Dirchlet's theorem is proven by continuous fractions.\\
    Say that $\frac{a}{b}$ is a record setting approximation to $\alpha$
    if $|\alpha - \frac{a}{b}| < |\alpha - \frac{c}{d}| \forall d < b$ and
    $\frac{a}{b} \not= \frac{c}{d}$.\\\\
    \underline{Example}: $\sqrt{2} \approx \frac{7}{5}$\\
    $\frac{6}{4}$, $\frac{4}{3}$, $\frac{3}{2}$, and $\frac{1}{1}$ are
    still good approximation.\\\\
    \begin{theorem}
      All record setting $\frac{a}{b}$ comes from continuous fraction expansion
      of $\alpha$.
    \end{theorem}
    A continuation fraction is denoted by $[a_{0}; a_1, \ldots, a_n]$ (finite) 
    or $[a_0; a_1, \ldots]$ (infinite).\\
    Both means that 
    $$
      [a_0] = a_0
    $$
    and 
    $$
      [a_0; a_1, \ldots] = a_0 + \frac{1}{[a_1, a_2, \ldots]}
    $$
    More rigorously, if you chop a continuous fraction to $c_k = [a_0, 
    a_1, \ldots, a_k]$.\\
    Then, $[a_0, a_1, \ldots]$ is by definition $\lim_{k \to \infty} c_k$.\\
    \begin{theorem}
      \begin{enumerate}
        \item $\lim_{k \to \infty} c_k$ always exists
        \item limits differ for two different continuous fractions
        \item Every $\alpha \in \mathbb{R}$ is reached.
      \end{enumerate}
    \end{theorem}
    Let's suppose all that.
    Take $\alpha \in \mathbb{R}$. Then,
    $$
      \alpha = [a_0, a_1, \ldots] = a_0 + \frac{1}{a_1 + \ldots}
    $$
    Say $\alpha = \sqrt{7} = 2.6\ldots$\\
    $a_0 = 2$\\
    \underline{Rules}: $a_i > 0$ for all $i$. In $[a_0, \ldots, a_n]$, $a_n
    >2$.\\
    In fact, this shows part 2 by induction.\\\\
    By this relation, if $\alpha = [a_0; a_1, a_2, \ldots]$, then
    $\frac{1}{1 - \alpha_0} = [a_1; a_2, \ldots]$\\
    Not a convergent. We chopped off head, not tail, so they're all unique.\\\\
    \begin{eqnarray*}
      \left\lfloor\sqrt{2} \right\rfloor &= & 1 \\
      \left\lfloor\frac{1}{\sqrt{2} - 1} \right\rfloor &= & 2 \\
      \left\lfloor\frac{1}{\frac{1}{\sqrt{2} - 1} - 2} \right\rfloor &= & 2 \\
    \end{eqnarray*}
    \begin{theorem}
      If $\alpha  = \sqrt{n}$, tne remainders in continued fraction expansion
      of $\alpha$ eventually reach $\sqrt{n} + k$ or $\frac{\sqrt{n} + k}{2}
      \Rightarrow$ continued fraction expansion repeats.
    \end{theorem}
    \begin{theorem}
      Converse is true. If continuous fraction expansion of $\alpha$ repeats,
      then $a$ is a root of $ax^2 + bx + c = 0$, $a,b,c \in \mathbb{Z}$.\\
      $\alpha = \frac{\sqrt{n} + d}{e}$
    \end{theorem}
    \begin{proof}
      Suppose $\alpha = \frac{a}{b}$ is rational, then by the theorem,
      continued fraction expansion of $\alpha$ terminates. In fact, it's
      just the euclidean algorithm to compute $gcd(a,b)$. Max(a,b) in 
      remainders keep decreasing.\\
      \begin{eqnarray*}
        \frac{a}{b} & = & q + \frac{r}{b} \\
          & = & \frac{1}{\frac{b}{r}} \\
      \end{eqnarray*}
      where $q = \lfloor\frac{a}{b}\rfloor$.\\
      Therefore, continued fraction expansion of $\frac{a}{b}$ terminates.
    \end{proof}
