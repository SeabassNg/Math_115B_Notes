\section*{1/9}
  \fbox{
    \begin{minipage}{7in}
      \underline{Theorem (Langrange)}: If $p$ is prime, $\alpha(x)$ is a 
      integer polynomial of degree $n$, then $\alpha(x) \equiv \mod p$ has $\le n$
      incongruent solutions. ($\alpha(x) \not= 0$)
    \end{minipage}
  }\\
  Compare with, In $\mathbb{Q}$ or $\mathbb{R}$, $\alpha(a) = 0$ has $\le n$
  solutions. In $\mathbb{C}$, $\alpha(x) = 0$ has exactly $n$ solutions counting
  repeats.\\
  \subsection*{Polynomials $\mod p$ in general}
    If we say that $\alpha(x) \equiv \beta(x) \mod n$, it could mean that
    the coefficients are equivalent or the values are congruent.\\
    They're not congruent situations. The former is the usual meaning and is
    stronger. The former implies the latter.\\
    \underline{Example}: $n=2$. Let's claim that $x^2 + x \equiv 0 \mod 2$\\
    There are only 2 cases, $0,1$.\\
    $0^2 + 0 \equiv 0$ and $1^2 + 1 \equiv 0$, so it's true.\\
    They're not equivalent as polynomials.
  \subsection*{Proof of Lagrange's theorem}
    Uses induction on degree of $a$ and Primality Lemma\\
    Primality Lemma - If $p | ab$, then $p | a$ or $p | b$.\\
    \underline{Base case}: Let deg($a$) = 0 where $a \not\equiv 0$.\\
    We want to know that $a$ has at most 0 roots $\mod p$.\\
    This is true because $a(x) = b \mod c$ where $b \not\equiv 0$ has no
    solution since no matter what $x$ is, $a(x) \not\equiv 0$ since it is
    $\equiv b$. \\
    \underline{Inductive case}: We have $a(x)$, say $x = r$ is $a \mod p$ roots.
    Otherwise, if there is no $r$, we're done.\\
    Then, $a(r) = b$, say I mean = in $\mathbb{Z}$, $b \equiv 0 \mod p$ or 
    $p | b$. We can change $a(x)$ to $\hat{a}(x) = a(x) - b$, doesn't change the 
    situation. $r$ is an ordinary root of $\hat{a}(x)$.\\
    $\hat{a}(x) = (x-r)c(x)$\\
    $a(x) \equiv (x-r)c(x) \mod p$.\\
    I claim that if $(s-r)c(s) \equiv 0 \mod p$ where $s \not= r$, then 
    $c(s) \equiv 0 \mod p$.\\
    Yes, this is true because $p \not| s - r$, so $p | c(s)$ or $\frac{1}{s-r}$
    exists $\mod p$.\\
    $a(x) \equiv (x-r)c(x)$, so deg($a$) = $n$, deg($c$) = n -1, $c(x)$ has
    $\le n-1$ roots.\\
    Claim says $a(x)$ has the roots that $c(x)$ does. $p$ is at most
    one more, so $a(x)$ has $\le n$ roots.\\\\
   \fbox{
    \begin{minipage}{7in}
      \underline{Theorem}: A polynomial with coefficients in any field factors
      uniquely into irreducible polynomials.
    \end{minipage}
  }\\\\
  Proof is similar to the unique factorication of integers last quarter. It's
  a bit more abstract, but the idea is still the same.\\\\
  Recall, we will care about $x^k \equiv 1 \mod p$.\\
  In general, $a(x) \equiv 0 \mod p$ is hard to solve when $p$ is big.\\
  The $\le$ deg($a$) solutions is only easy part.\\
  What about $a(x) \equiv 0 \mod n$ where $n$ is not prime?\\
  \underline{Example}: $x^3 + x + 5 = 0 \mod 21$\\
  From the chinese remainder theorem, $a(x) \equiv 0 \mod n \Leftrightarrow
  a(x) \equiv 0 \mod \text{prime power factors of $n$}$\\
  That still leaves $a(x) \equiv 0 \mod p^k$.\\
  \underline{Note}: mod 49 $\not\Leftrightarrow$ mod 7, BUT mod 49 $\implies$
  mod 7.\\
  So, from our example, $x^3 + x + 5 \equiv 0 \mod 49 \implies x^3 + x + 5
  \equiv 0 \mod 7$.\\
  So, our plan: solve $\mod p^{k-1}$. First say $x \equiv r \mod p^k$. Then,
  lift $x \equiv r + tp^{k-1} \mod p^k$. $t$ exists $\mod p$. This is called
  "Hensel lifting".\\\\
   \fbox{
    \begin{minipage}{7in}
      \underline{Theorem(Hensel)}: the equation for $t$ is linear and uses 
      $\frac{d}{dx} a(x)$.
    \end{minipage}
  }\\\\

