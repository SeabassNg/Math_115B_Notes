\section*{3/13}
  \subsection*{Gaussian integers}
    \begin{definition}
      If $u \in \mathbb{Z}[i]$ has a reciprocal $u^{-1} \in \mathbb{Z}[i]$, 
      then it's a unit.
    \end{definition}
    So, which elements is that?
    \begin{definition}
      If $z \in \mathbb{Z}[i]$, $N(z) = |z|^2 = x^2 + y^2 \in \mathbb{Z}$
      where $z = x + iy$.
    \end{definition}
    \underline{Note}: $N(z)$ is the number theorists norm and $|z|$ is
      the analyst norm.\\
    \underline{Note}: $N(z) = 0 \Leftrightarrow z = 0$.\\
    \underline{Note}: $z | N(z)$ as Gaussian integers because $N(z) = z \cdot
      \overline{z} = (x + iy)(x - iy)$.\\
    \underline{Note}: $z \in \mathbb{Z}[i] \Leftrightarrow \overline{z} \in
      \mathbb{Z}[i]$.\\
    \underline{Note}: $N(z_1z_2) = N(z_1)N(z_2)$
    \begin{theorem}[Unique Factorization of Gaussian integers]
      $\mathbb{Z}[i]$ have unique factorization into Gaussian primes.
    \end{theorem}
    In fact, if $u$ is a unit, $z \approx uz$. $z$ and $uz$ are associates.\\\\
    Associate primes are equivalent in the sense of factorization.\\
    \underline{Example}: Is 2 square free in $\mathbb{Z}[i]$?\\
      $2 = (1 + i)(1 - i)$\\
      Are there repeats? Yes! $(1 - i) = -i(1 + i)$\\
      Therefore, $2 = i(1 + i)^2$\\
      Therefore, 2 is not square free.\\\\
    Why are we caring about units? In the past, we can just restrict everything
    to positive numbers. Now, we can't.\\
  \subsection*{Why $\mathbb{Z}$ has unique factorization?}
    For uniqueness, division with remainders, which led to euclidean algorithm,
    which led to primality lemma, which led to unique factorization.\\
    All of our implications are inevitably provided, you get it started.\\\\
    As for the existence, you have this if elements have a size which decreases
    when you factor.\\\\
    Does $N(z)$ have a positive integer size such that if $z = ab$ where
    $a,b$ are not units, so that $N(a), N(b) < N(z)?$ Yes!\\
    $N(z) = N(a)N(b)$ and $N(a),N(b) \ge z$.\\\\
    \underline{Example}: $2 = (1 + i)(1 - i)$ where $N(2) = 4$ and $N(1+i) = 
    2$.\\
  \subsection*{Division with remainders}
    We want the following: Given $a,b \in \mathbb{Z}[i]$, $a = qb + r$ such 
    that $N(r) < N(b)$.\\
    In the Cartesian coordinates, multiplying by $b$ dilates by $|b|$ and
    rotates by the angle of $b$, which is known as $Arg(b)$.\\\\
    \underline{Example}: $b = 2 + i$.
