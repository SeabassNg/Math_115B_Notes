\section*{3/4}
  $\alpha \approx \frac{a}{b}$ is "best" means that $\alpha - \frac{a}{b}| >
  |\alpha - \frac{c}{d}| \Rightarrow d > b$.\\
  \begin{theorem}
    If $C_k = \frac{s_k}{t_k}$ is a convergent of $\alpha$, then it is the
    best. In fact, if there exists $\frac{a}{b}$ is better, $b \ge t_{k+1}$\\
  \end{theorem}
  An even better theorem
  \begin{theorem}
    $|t_k \alpha - s_k| > |b\alpha - a| \Rightarrow b \ge t_{k +1 }$\\
    If $b < s_k$ and $|t_k\alpha - s_k| > |b\alpha - a|$, then
    this is weaker than $\frac{a}{b}$ being closer
  \end{theorem}
  \subsection*{Periodic continued fractions}
    Suppose $\alpha = [a_0; a_1, a_2, \ldots, \overline{b_1, \ldots, b_l}]$.\\
    Then, what can $\alpha$ be? The right side does converge can use this solve
    for $\alpha$.\\
    \begin{lemma}
      $f(x) = [a_0; a_1, \ldots, x]$ is a fractional linear transformation.
    \end{lemma}
    \begin{proof}
      $$
        a_k + \frac{1}{x} = \frac{a_kx + 1}{bx + 0}
      $$
    \end{proof}
    Is 1 over a fractional linear transformation a fractional linear 
    transformation? Yes\\
    Is $a + $ fractional linear transformation a fractional linear 
    transformation? Yes\\
    \begin{eqnarray*}
      a + \frac{jx + k}{lx + m} & = & \frac{a(lx + m) + jx + k}{lx + m}\\
        & = & \frac{x (al + j) + (am + k)}{ lx + m}
    \end{eqnarray*}
    \underline{Case 1}: $\alpha = [b_0; b_1, \ldots, b_l]$, so
    $$
      \alpha = b_0 + \frac{1}{b_1 + \frac{1}{1 + \vdots + \frac{1}{b_l + 
      \frac{1}{\alpha}}}} = \frac{a \alpha + b}{c \alpha + d}
    $$
    by the lemma.\\
    $\alpha$ is a root of $x = \frac{ax + b}{cx+d}$, which is approximate to
    a quadratic equation.\\\\
    \underline{General case}:
    If $[b_0; b_1, \ldots, b_l] = \frac{a + b\sqrt{n}}{c}$.\\
    Let $\alpha = [a_0, a_1, \ldots, a_k, \frac{a + b \sqrt{n}}{c}]$.\\
    \begin{lemma}
      $\alpha$ is then also a quadratic irrational.
    \end{lemma}
    \begin{proof}
      $\frac{a + b\sqrt{n}}{c}$ is a quadratic irrational.\\
      Then, is $d + $ a quadratic irrational $ = $ a quadratic irrational?\\
      $$
        d + \frac{a + b\sqrt{n}}{c} = \frac{(a+cd) + b\sqrt{n}}{c}
      $$
      Is $\frac{1}{\text{quadratic irrational}} = 
        \text{quadratic irrational}$?\\
      $$
        \frac{c}{a + b\sqrt{n}} = \frac{(a - b\sqrt{n})c}{a^2 - b^2n}
      $$
      Yes!
      Therefore, by induction...
    \end{proof}
  \subsection*{Diophantine equations}
    Have some polynomial equations, in some variables.\\
    $x^2 + y^2 = z^2$, $x^2 + y^2 = z$, $x^3 + y^3 = z^3$, $x^2 + y^2 + z^2 = 
    w$, $x^2 + y^2 + z^2 + w^2 = t$\\\\
    \begin{theorem}
      It is impossible to analyze a general Diophantine equation.
    \end{theorem}
    \begin{proof}
      The difficulty of this problem is too much for us.
    \end{proof}
    \begin{theorem}
      If $A$ is a formal math conjecture, then there exists a polynomial,
      $p_A(x_1, \ldots x_k) = 0$, computable directly from $A$ that has 
      solutions in $\mathbb{Z}$ iff A is true.
    \end{theorem}
    \begin{theorem}
      If $A$ is a formal computer program, then there exists a $p_A(x_1, \ldots
      x_n)$ whose positive values, $x_1, \ldots, x_n \in \mathbb{Z}$ are
      the output of $A$.
    \end{theorem}
    We have $x^2 + y^2 = z^2$ where $x, y, z \in \mathbb{Z}$. These are called
    Pythagorean triples.\\
    Without loss of generality, let's just consider $x,y,z > 0$.\\
    We want $gcd(x,y,z) = 1$ because otherwise, this is just trivial. The
    triples with $gcd(x,y,z) = 1$ are called \underline{primitives}.\\\\
    \underline{A plan}: 
    $$
      (n+1)^2 - n^2 = 2n + 1
    $$
    sometimes that odd number on the right are squares.\\
    \underline{Examples}:
    $4^2 + 3^2 + 5^2$\\
    $5^2 + 5^2 + 13^2$\\
    $24^2 + 7^2 + 25^2$\\
    $40^2 + 9^2 + 41^2$\\\\
    \underline{General formula}:
      $x = m^2 - n^2$, $y = 2mn$, and $z = m^2 + n^2$ where $gcd(m,n) = 1$
